\documentclass[12pt,a4paper]{article}
\usepackage[utf8]{inputenc}
\usepackage[spanish]{babel}
\usepackage{amsmath,amsfonts,amssymb}
\usepackage{graphicx}
\usepackage{hyperref}
\usepackage{xcolor}
\usepackage{listings}
\usepackage{booktabs}
\usepackage{float}
\usepackage{enumitem}
\usepackage{geometry}
\usepackage{titlesec}

\geometry{margin=2.5cm}

\hypersetup{
    colorlinks=true,
    linkcolor=blue,
    filecolor=magenta,      
    urlcolor=cyan,
    pdftitle={Machine Learning y Deep Learning para Ingeniería Pesquera},
    pdfauthor={Tu Nombre},
}

\lstset{
    language=Python,
    basicstyle=\ttfamily\small,
    keywordstyle=\color{blue},
    commentstyle=\color{green!60!black},
    stringstyle=\color{red},
    numbers=left,
    numberstyle=\tiny\color{gray},
    numbersep=5pt,
    backgroundcolor=\color{gray!10},
    frame=single,
    tabsize=4,
    captionpos=b,
    breaklines=true,
    breakatwhitespace=false,
    showspaces=false,
    showstringspaces=false,
    showtabs=false,
    morekeywords={import, from, class, def, for, while, if, else, elif, try, except, finally, with, as, return, yield}
}

\titleformat{\section}
  {\normalfont\Large\bfseries}{\thesection}{1em}{}
\titleformat{\subsection}
  {\normalfont\large\bfseries}{\thesubsection}{1em}{}

\title{\textbf{Machine Learning \& Deep Learning para Ingeniería Pesquera}\\
\large Marco Teórico y Aplicaciones Prácticas}
\author{Tu Nombre}
\date{\today}

\begin{document}

\maketitle
\tableofcontents
\newpage

\section*{Introducción}
\addcontentsline{toc}{section}{Introducción}

Este compendio representa una colección estructurada de técnicas de Ciencia de Datos, Inteligencia Artificial y Analítica Avanzada aplicadas específicamente al sector pesquero y acuícola argentino. El objetivo es proporcionar a ingenieros pesqueros, biólogos marinos, gestores de recursos y otros profesionales del sector, herramientas computacionales modernas que permitan optimizar procesos, predecir comportamientos y tomar decisiones basadas en datos.

La pesca y la acuicultura en Argentina enfrentan desafíos únicos debido a la extensión de su plataforma continental, la diversidad de especies comerciales y las complejas dinámicas oceanográficas. Las técnicas de Machine Learning y Deep Learning ofrecen soluciones innovadoras para abordar estos desafíos, desde la predicción de zonas de pesca hasta la optimización de cadenas de suministro y el monitoreo de la sostenibilidad de los recursos.

\subsection*{Objetivos Generales}

\begin{enumerate}
    \item Proporcionar herramientas de IA para optimización de procesos pesqueros
    \item Modelado predictivo para gestión sostenible de recursos
    \item Automatización de análisis mediante visión computacional y sensores
    \item Facilitar la toma de decisiones basada en datos en el sector pesquero
    \item Promover prácticas sostenibles mediante el análisis avanzado de datos
\end{enumerate}

\subsection*{Estructura del Compendio}

Este manual está organizado en secciones temáticas que abarcan desde técnicas básicas de Machine Learning hasta métodos avanzados de Deep Learning. Cada sección incluye:

\begin{itemize}
    \item \textbf{Fundamento teórico}: Explicación de los conceptos matemáticos y estadísticos subyacentes
    \item \textbf{Aplicación práctica}: Casos de uso específicos para el sector pesquero argentino
    \item \textbf{Implementación en código}: Ejemplos ejecutables en Python con bibliotecas estándar
    \item \textbf{Vinculación con fuentes de datos reales}: Referencias a repositorios oficiales (INIDEP, MAGYP)
    \item \textbf{Consideraciones éticas y limitaciones}: Discusión sobre el alcance y restricciones de cada técnica
\end{itemize}

Cada notebook está diseñado para ser autocontenido, pero se recomienda seguir el orden propuesto para una comprensión progresiva de las técnicas.

\newpage
\section{Aprendizaje Supervisado para Clasificación en Pesquerías}

\subsection{Fundamentos Teóricos}

El aprendizaje supervisado es una rama del Machine Learning donde el algoritmo aprende a partir de datos etiquetados. En el contexto pesquero, esto puede aplicarse a problemas como la clasificación de especies, la determinación de la viabilidad de proyectos acuícolas, o la identificación de buques eficientes.

Los algoritmos de clasificación más relevantes para el sector incluyen:

\begin{itemize}
    \item \textbf{Regresión Logística}: Modelo probabilístico lineal para clasificación binaria
    \item \textbf{Random Forest}: Conjunto de árboles de decisión para problemas complejos
    \item \textbf{Support Vector Machines (SVM)}: Clasificador de margen máximo
    \item \textbf{K-Nearest Neighbors (KNN)}: Clasificación basada en similitud
    \item \textbf{Naive Bayes}: Clasificador probabilístico basado en el teorema de Bayes
\end{itemize}

\subsection{Aplicaciones en Ingeniería Pesquera}

\subsubsection{Notebooks vinculados}
\begin{itemize}
    \item \textbf{Regresión Logística}: Viabilidad de proyectos de acuicultura
    \item \textbf{Random Forest}: Procesamiento de productos pesqueros
    \item \textbf{SVM}: Clasificación de capturas
    \item \textbf{KNN}: Identificación de buques eficientes
    \item \textbf{Naive Bayes}: Identificación de especies
\end{itemize}

\subsection{Caso de Estudio: Viabilidad de Proyectos Acuícolas}

La acuicultura marina en Argentina presenta un gran potencial, especialmente en regiones como Tierra del Fuego. La regresión logística permite modelar la probabilidad de éxito de un proyecto acuícola basándose en variables ambientales críticas como temperatura, salinidad y oxígeno disuelto.

El notebook \texttt{Regresion\_Log\_Viabilidad\_Proyecto\_Acuicultura.ipynb} implementa un modelo que clasifica sitios potenciales como "viables" o "no viables" basándose en parámetros ambientales y operativos.

\section{Aprendizaje Supervisado para Regresión en Pesquerías}

\subsection{Fundamentos Teóricos}

Los algoritmos de regresión permiten predecir valores continuos, lo que resulta fundamental para estimar parámetros como biomasa, rendimiento productivo o crecimiento de especies cultivadas. A diferencia de la clasificación, la regresión proporciona estimaciones numéricas precisas.

Técnicas principales:
\begin{itemize}
    \item \textbf{Regresión Lineal y Polinomial}: Para relaciones simples entre variables
    \item \textbf{Random Forest Regressor}: Para capturar relaciones no lineales complejas
    \item \textbf{Support Vector Regression (SVR)}: Para problemas con alta dimensionalidad
\end{itemize}

\subsection{Aplicaciones en Ingeniería Pesquera}

\subsubsection{Notebooks vinculados}
\begin{itemize}
    \item \textbf{Random Forest Regressor}: Predicción de estadios larvales en crustáceos
    \item \textbf{SVR}: Modelado del crecimiento en salmones
\end{itemize}

\subsection{Caso de Estudio: Predicción de Desarrollo Larval}

El desarrollo larval de crustáceos decápodos es un proceso crítico en acuicultura que depende de múltiples variables ambientales. El notebook \texttt{RandomForest\_Reg\_estadio\_larval.ipynb} implementa un modelo que predice el estadio de desarrollo larval basándose en parámetros como temperatura, salinidad, pH y alimentación.

\section{Aprendizaje No Supervisado para Análisis de Patrones}

\subsection{Fundamentos Teóricos}

El aprendizaje no supervisado trabaja con datos no etiquetados para descubrir patrones, agrupaciones o relaciones ocultas. En el sector pesquero, estas técnicas son valiosas para segmentar flotas, identificar patrones de co-captura o reducir la dimensionalidad de datos complejos.

Técnicas principales:
\begin{itemize}
    \item \textbf{Análisis de Componentes Principales (PCA)}: Reducción de dimensionalidad
    \item \textbf{Algoritmos de Clustering (K-means, DBSCAN)}: Agrupación de datos similares
    \item \textbf{Reglas de Asociación (Apriori)}: Descubrimiento de patrones frecuentes
\end{itemize}

\subsection{Aplicaciones en Ingeniería Pesquera}

\subsubsection{Notebooks vinculados}
\begin{itemize}
    \item \textbf{PCA}: Análisis de características de flota pesquera
    \item \textbf{Apriori}: Patrones de co-captura en desembarques
\end{itemize}

\subsection{Caso de Estudio: Segmentación de Flota Pesquera}

La heterogeneidad de la flota pesquera argentina requiere un análisis multivariado para su caracterización. El notebook \texttt{PrincipalComponentAnalysis\_Flota.ipynb} implementa PCA para reducir la dimensionalidad de variables como eslora, potencia, capacidad de bodega y antigüedad, permitiendo identificar segmentos homogéneos de embarcaciones.

\section{Deep Learning para Análisis de Imágenes y Sensores}

\subsection{Fundamentos Teóricos}

Las redes neuronales profundas han revolucionado el procesamiento de imágenes y datos de sensores. En el sector pesquero, estas técnicas permiten automatizar tareas como la detección de embarcaciones en imágenes satelitales, el conteo de larvas o el monitoreo de parámetros ambientales.

Arquitecturas relevantes:
\begin{itemize}
    \item \textbf{Redes Neuronales Convolucionales (CNN)}: Para procesamiento de imágenes
    \item \textbf{Redes Neuronales Recurrentes (RNN/LSTM)}: Para series temporales
\end{itemize}

\subsection{Aplicaciones en Ingeniería Pesquera}

\subsubsection{Notebooks vinculados}
\begin{itemize}
    \item \textbf{CNN}: Predicción de distribución de flotas pesqueras mediante imágenes satelitales
    \item \textbf{OpenCV}: Contador automático de larvas
\end{itemize}

\subsection{Caso de Estudio: Monitoreo Satelital de Flotas}

El seguimiento de flotas pesqueras mediante imágenes satelitales es fundamental para la gestión y control de la actividad pesquera. El notebook \texttt{CNNs\_PrediccionSatelitalDistFlotaPesquera.ipynb} implementa una CNN para detectar embarcaciones en imágenes satelitales y predecir su distribución espacial futura.

\begin{lstlisting}[caption={Ejemplo simplificado de CNN para imágenes satelitales}]
# Ejemplo simplificado de CNN para imágenes satelitales
from tensorflow.keras.models import Sequential
from tensorflow.keras.layers import Conv2D, MaxPooling2D, Flatten, Dense

model = Sequential([
    Conv2D(32, (3,3), activation='relu', input_shape=(256,256,3)),
    MaxPooling2D(2,2),
    Conv2D(64, (3,3), activation='relu'),
    MaxPooling2D(2,2),
    Flatten(),
    Dense(128, activation='relu'),
    Dense(1, activation='sigmoid')
])
\end{lstlisting}

\section{Modelado Causal y Redes Bayesianas}

\subsection{Fundamentos Teóricos}

Las redes bayesianas permiten modelar relaciones causales entre variables, lo que resulta especialmente valioso para comprender sistemas complejos como los ecosistemas marinos o las pesquerías. Estas redes representan dependencias probabilísticas mediante grafos dirigidos acíclicos.

Conceptos clave:
\begin{itemize}
    \item \textbf{Probabilidad condicional}: $P(A|B)$
    \item \textbf{Teorema de Bayes}: $P(A|B) = \frac{P(B|A)P(A)}{P(B)}$
    \item \textbf{Independencia condicional}: Variables que son independientes dado un conjunto de otras variables
\end{itemize}

\subsection{Aplicaciones en Ingeniería Pesquera}

\subsubsection{Notebooks vinculados}
\begin{itemize}
    \item \textbf{Redes Bayesianas}: Modelado causal para sostenibilidad pesquera
\end{itemize}

\subsection{Caso de Estudio: Sostenibilidad de Pesquerías Marinas}

La sostenibilidad de las pesquerías depende de múltiples factores interrelacionados. El notebook \texttt{BayesianNetworks\_SostenibilidadPesquera.ipynb} implementa una red bayesiana que modela las relaciones causales entre variables ambientales, operativas y económicas para identificar los factores críticos que afectan la sostenibilidad.

\begin{lstlisting}[caption={Ejemplo de Red Bayesiana para sostenibilidad pesquera}]
# Ejemplo de Red Bayesiana para sostenibilidad pesquera
from pgmpy.models import BayesianModel

modelo = BayesianModel([
    ('Temperatura', 'Biomasa'),
    ('Presion_Pesquera', 'Biomasa'),
    ('Biomasa', 'Sostenibilidad'),
    ('Regulacion', 'Presion_Pesquera'),
    ('Costos_Operativos', 'Presion_Pesquera')
])
\end{lstlisting}

\section{Graph Neural Networks para Cadenas de Suministro}

\subsection{Fundamentos Teóricos}

Las Graph Neural Networks (GNNs) extienden las capacidades del deep learning a datos estructurados como grafos. En el sector pesquero, estas técnicas son ideales para modelar redes logísticas, cadenas de suministro o relaciones entre puertos y centros de procesamiento.

Conceptos clave:
\begin{itemize}
    \item \textbf{Representación de grafos}: Nodos, aristas y sus atributos
    \item \textbf{Propagación de mensajes}: Cómo la información fluye entre nodos conectados
    \item \textbf{Embeddings de nodos}: Representaciones vectoriales de nodos que capturan su posición en el grafo
\end{itemize}

\subsection{Aplicaciones en Ingeniería Pesquera}

\subsubsection{Notebooks vinculados}
\begin{itemize}
    \item \textbf{GNN}: Modelado y análisis de la cadena de suministro pesquero
\end{itemize}

\subsection{Caso de Estudio: Optimización de Cadena de Suministro}

La cadena de suministro pesquero en Argentina involucra múltiples puertos, rutas de transporte y centros de procesamiento. El notebook \texttt{GNNs\_CadenaSuministro\_Pesca.ipynb} implementa una GNN para modelar esta red compleja, identificar cuellos de botella y optimizar la distribución de recursos.

\section{Aprendizaje por Refuerzo para Gestión Pesquera}

\subsection{Fundamentos Teóricos}

El aprendizaje por refuerzo permite a los agentes aprender a tomar decisiones secuenciales mediante la interacción con un entorno. En la gestión pesquera, estas técnicas pueden modelar estrategias óptimas de captura, políticas de cuotas o planes de manejo adaptativo.

Enfoques principales:
\begin{itemize}
    \item \textbf{Model-Free RL}: Aprende directamente de la experiencia sin modelar el entorno
    \item \textbf{Model-Based RL}: Aprende un modelo del entorno para planificar acciones
\end{itemize}

\subsection{Aplicaciones en Ingeniería Pesquera}

\subsubsection{Notebooks vinculados}
\begin{itemize}
    \item \textbf{Model-Based RL}: Aprendizaje del modelo para cuotas de captura sostenible
\end{itemize}

\subsection{Caso de Estudio: Determinación de Cuotas Sostenibles}

La determinación de cuotas de captura que maximicen el rendimiento económico mientras garantizan la sostenibilidad del recurso es un problema complejo. El notebook \texttt{Learn\_the\_Model\_CuotaCapturaSostenible.ipynb} implementa un enfoque de RL basado en modelo para aprender políticas óptimas de captura.

\section{Cuadro Integrador de Técnicas y Aplicaciones}

\begin{table}[H]
\centering
\begin{tabular}{@{}llll@{}}
\toprule
\textbf{Problema Pesquero} & \textbf{Técnica ML/DL} & \textbf{Caso de Estudio} & \textbf{Notebook} \\ \midrule
Viabilidad Acuícola & Regresión Logística & Proyectos en Tierra del Fuego & \texttt{Regresion\_Log\_Viabilidad...} \\
Procesamiento de Productos & Random Forest & Calidad de productos & \texttt{Random\_Forest\_Procesamiento...} \\
Clasificación de Capturas & SVM & Muestreo y clasificación & \texttt{SVM\_Clas\_captura.ipynb} \\
Eficiencia de Buques & KNN & Flota argentina & \texttt{KNN\_buque\_eficiente.ipynb} \\
Identificación de Especies & Naive Bayes & Especies comerciales & \texttt{Naive\_Bayes\_Identi\_Especie...} \\
Desarrollo Larval & Random Forest Regressor & Crustáceos decápodos & \texttt{RandomForest\_Reg\_estadio...} \\
Crecimiento en Cultivo & SVR & Salmónidos en sistemas RAS & \texttt{SVM\_Salmones.ipynb} \\
Segmentación de Flota & PCA & Flota argentina & \texttt{PrincipalComponentAnalysis...} \\
Patrones de Co-captura & Apriori & Desembarques en puertos & \texttt{AprioriAlgorithm\_Co-captura...} \\
Monitoreo de Flotas & CNN & ZEE Argentina - Imágenes & \texttt{CNNs\_PrediccionSatelital...} \\
Conteo de Larvas & OpenCV & Laboratorios de acuicultura & \texttt{Contador.ipynb} \\
Sostenibilidad Pesquera & Redes Bayesianas & Pesquerías marinas & \texttt{BayesianNetworks\_Sostenibilidad...} \\
Cadena de Suministro & GNN & Logística pesquera & \texttt{GNNs\_CadenaSuministro\_Pesca...} \\
Cuotas de Captura & Model-Based RL & Manejo adaptativo & \texttt{Learn\_the\_Model\_CuotaCaptura...} \\ \bottomrule
\end{tabular}
\caption{Integración de técnicas ML/DL con problemas pesqueros y notebooks correspondientes}
\label{tab:integracion}
\end{table}

\section{Fuentes de Datos y Referencias}

\subsection{Fuentes Institucionales}

\begin{itemize}
    \item \textbf{INIDEP (Instituto Nacional de Investigación y Desarrollo Pesquero)}
    \begin{itemize}
        \item Informes técnicos y científicos
        \item Estadísticas pesqueras
        \item Estudios biológicos de especies comerciales
        \item \url{https://www.argentina.gob.ar/inidep}
    \end{itemize}
    
    \item \textbf{Ministerio de Agricultura, Ganadería y Pesca}
    \begin{itemize}
        \item Estadísticas de desembarques
        \item Informes sobre la evolución de la pesca marina
        \item \url{https://www.magyp.gob.ar/sitio/areas/pesca_maritima/}
    \end{itemize}
\end{itemize}

\subsection{Publicaciones Científicas}

\begin{itemize}
    \item \textbf{Revistas especializadas}
    \begin{itemize}
        \item Fisheries Research
        \item Aquaculture
        \item Revista de Biología Marina
        \item Latin American Journal of Aquatic Research
    \end{itemize}
    
    \item \textbf{Estudios de referencia}
    \begin{itemize}
        \item "Estadísticas de la pesca marina en la Argentina: Evolución de los desembarques 2012–2016"
        \item "Acuicultura en Argentina: red de actores, procesos de producción y espacios para el agregado de valor"
        \item "Aspectos del desove y fecundidad del langostino Pleoticus muelleri" (Macchi et al., 1992)
    \end{itemize}
\end{itemize}

\subsection{Repositorios de Datos}

\begin{itemize}
    \item \textbf{Datos satelitales}
    \begin{itemize}
        \item Copernicus Marine Service
        \item NASA Earth Data
        \item NOAA Fisheries
    \end{itemize}
    
    \item \textbf{Datos oceanográficos}
    \begin{itemize}
        \item Sistema Nacional de Datos del Mar (SNDM)
        \item World Ocean Database
    \end{itemize}
\end{itemize}

\section{Consideraciones Éticas y Limitaciones}

\subsection{Ética en el Uso de Datos Pesqueros}

\begin{itemize}
    \item \textbf{Privacidad}: Los datos de operaciones pesqueras pueden contener información sensible sobre rutas comerciales o zonas de pesca. Es fundamental anonimizar adecuadamente estos datos.
    
    \item \textbf{Transparencia}: Los modelos predictivos deben ser transparentes en cuanto a sus limitaciones y márgenes de error, especialmente cuando se utilizan para la toma de decisiones que afectan a comunidades pesqueras.
    
    \item \textbf{Equidad}: Los beneficios de la aplicación de estas tecnologías deben distribuirse equitativamente entre los diferentes actores del sector pesquero, desde grandes empresas hasta pescadores artesanales.
\end{itemize}

\subsection{Limitaciones Técnicas}

\begin{itemize}
    \item \textbf{Calidad de datos}: Muchos datasets pesqueros presentan problemas de completitud, consistencia o sesgo que pueden afectar el rendimiento de los modelos.
    
    \item \textbf{Complejidad de los ecosistemas}: Los sistemas marinos son altamente complejos y dinámicos, lo que dificulta su modelado preciso mediante técnicas de ML/DL.
    
    \item \textbf{Validación}: La validación de modelos predictivos en contextos pesqueros puede requerir largos períodos de tiempo debido a la estacionalidad y variabilidad natural de los recursos.
    
    \item \textbf{Interpretabilidad}: Algunos modelos avanzados (como deep learning) funcionan como "cajas negras", lo que puede limitar su aceptación en contextos donde la interpretabilidad es crucial para la toma de decisiones.
\end{itemize}

\section{Conclusiones y Perspectivas Futuras}

\subsection{Síntesis}

Este compendio presenta un panorama integral de las aplicaciones de Machine Learning y Deep Learning en el sector pesquero y acuícola argentino. Desde técnicas básicas de clasificación hasta métodos avanzados como redes bayesianas y GNNs, estas herramientas ofrecen soluciones innovadoras para los desafíos que enfrenta el sector.

La integración de datos provenientes de diversas fuentes (satelitales, oceanográficos, biológicos, económicos) permite un enfoque holístico para la gestión de recursos pesqueros, promoviendo prácticas más sostenibles y eficientes.

\subsection{Tendencias Emergentes}

\begin{itemize}
    \item \textbf{Integración de múltiples fuentes de datos}: Combinación de datos satelitales, sensores in-situ, información biológica y económica.
    
    \item \textbf{Modelos híbridos}: Fusión de conocimiento experto con técnicas de ML/DL para mejorar la interpretabilidad y precisión.
    
    \item \textbf{Sistemas de alerta temprana}: Desarrollo de sistemas predictivos para fenómenos como floraciones algales nocivas o cambios en la distribución de especies.
    
    \item \textbf{Democratización de herramientas}: Creación de interfaces y aplicaciones que permitan a usuarios no técnicos beneficiarse de estas tecnologías.
\end{itemize}

\subsection{Próximos Pasos}

Este compendio es un proyecto en evolución que se enriquecerá con:

\begin{itemize}
    \item Incorporación de nuevos casos de estudio basados en datos reales
    \item Desarrollo de interfaces interactivas para visualización de resultados
    \item Colaboraciones con instituciones del sector para validación de modelos
    \item Extensión a nuevas técnicas emergentes como federated learning o neuro-symbolic AI
\end{itemize}

El objetivo final es crear un ecosistema de herramientas computacionales que contribuyan a la sostenibilidad y eficiencia del sector pesquero argentino, promoviendo una gestión basada en datos y evidencia científica.

\end{document}